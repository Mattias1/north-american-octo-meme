\documentclass{article}

\title{Simulation assignment}
\author{Chiel ten Brinke (3677133) and Mattias Beimers (3672565)}
\date{\vspace{-3ex}}

\usepackage{amsmath}
\usepackage{amssymb}
\usepackage{amsthm}
\usepackage{enumerate}
\usepackage{caption}
\usepackage{subcaption}
\usepackage{tikz}
\usepackage{float}
\usepackage[margin=1.75in]{geometry}
\usepackage[parfill]{parskip}

\usetikzlibrary{arrows,shapes,decorations.pathmorphing}

\newcommand{\Break}{\State \textbf{break}}
\renewcommand{\qed}{\hfill \ensuremath{\square}}

\newcommand{\calC}{\ensuremath{\mathcal{C}}}
\newcommand{\calCC}{\ensuremath{\mathcal{C'}}}
\newcommand{\reduc}{\ensuremath{\leq_P} }

\newtheorem{lemma}{Lemma}[section]

\begin{document}
    \tikzstyle{vertex}=[circle,very thick,draw=black,fill=black!25,minimum size=35pt,inner sep=0pt,align=center,text width=30pt,font=\scriptsize]
    \tikzstyle{fakeVertex}=[circle,minimum size=1pt,inner sep=0pt]
    \tikzstyle{edge} = [draw,thick,-]
    \tikzstyle{arc} = [draw,thick,->]
    \tikzstyle{weight} = [font=\scriptsize]

    \maketitle

    
    \section{Problem description}
        We present a simulation study for the buffer- and batch-size design of a production line of DVD's.
        Moreover, we analyze improvements for the line, by, among others, finding the best buffer- and bacth-sizes and alleviating the impact of bottlenecks.

        The production process of DVD's consists of the following steps:
        \begin{enumerate}[A.]
            \item Injection molding
            \item Dye coating and drying
            \item Sputtering, lacquer coating, drying
            \item Printing and finishing 
        \end{enumerate}
        The machines have limited buffer space, and step C is done with batch processing.
        Machines can break down, possibly causing them to be out of production for a while,
        and discarding DVD's they were producing. Also some machines need cleaning or replacing of ink.
        These problems are solved by repairmen, working in shifts.

    \section{Assumptions}
        These are some of our assumptoins:
        \begin{enumerate}
            \item 
                Machine A discards any DVD's that is being produced at the event of a breakdown.
            \item 
                Machines C and D are expected to start with cleaning or ink replacing after the (batch of) DVD's is finished.
        \end{enumerate}

    \section{Analysis of the problem}
        We want to solve this problem by perfomring a simulation study. % ???
        

    \section{The model}
    \subsection{The eventgraph}
        This is a part of our event graph (without the breakdowns).
        \begin{figure}[H]
            \centering
            \begin{tikzpicture}[auto,swap]
                % The machine start and finish events
                \node[vertex] (sA) at (-2,0) {MA start};
                \node[vertex] (fA) at (1,0) {MA finish};
                \node[vertex] (sB) at (-2,-2) {MB start};
                \node[vertex] (fB) at (1,-2) {MB finish};
                \node[vertex] (sC) at (-2,-5) {MC start};
                \node[vertex] (fC) at (1,-5) {MC finish};
                \node[vertex] (sD) at (-2,-7) {MD start};
                \node[vertex] (fD) at (1,-7) {MD finish};

                % Buffer events
                \node[vertex] (assembly) at (-0.5,-3.5) {Put in crate};

                % The breakdown events
                \node[fakeVertex] (sos3) at (-5,-2.2) {};
                \node[vertex] (rfA) at (-4,1) {MA finishes repairing};
                \node[vertex] (rsA) at (-4.5,-1.1) {MA starts repairing};
                \node[vertex] (rfC) at (-4,-4.6) {MC finishes cleaning};
                \node[vertex] (rsC) at (-4,-6.6) {MC starts cleaning};
                \node[vertex] (rfD) at (-0.5,-8.5) {MD finishes ink replacing};

                % The start- and end- of simulation events
                \node[fakeVertex] (sos1) at (-2.5,-1) {};
                \node[fakeVertex] (sos2) at (3,-7.5) {};
                \node[vertex] (eos) at (3.5,-6.5) {End of sim};

                % Arcs for machine start and finish events
                \path[arc,bend left=10] (sA) to (fA);
                \path[arc,bend left=10,dashed] (fA) to (sA);
                \path[arc,dashed] (fA) to (sB); % buffer.add_product
                \path[arc,bend left=10] (sB) to (fB);
                \path[arc,bend left=10,dashed] (fB) to (sB);
                \path[arc,bend left=10] (sC) to (fC);
                \path[arc,bend left=10,dashed] (fC) to (sC);
                \path[arc,dashed] (fC) to (sD); % buffer.add_product
                \path[arc,bend left=10] (sD) to (fD);
                \path[arc,bend left=10,dashed] (fD) to (sD);
                %\path[arc] (fD) to (eos);

                % Arcs for buffer events
                \path[arc,dashed] (sB) to (sA); % buffer.remove_product
                \path[arc,dashed] (fB) to (assembly); % buffer.add_product
                \path[arc,bend left=90,dashed,radius=30] (assembly) to (assembly);
                % TODO: RADIUS
                \path[arc,dashed] (assembly) to (sC); % buffer.add_product
                \path[arc,dashed] (sC) to (sB); % buffer.remove_product
                \path[arc,dashed] (sD) to (sC); % buffer.remove_product

                % Arcs for breakdown events
                \path[arc,bend left=15] (rsA) to (rfA);
                \path[arc,bend left=15,dashed] (rfA) to (rsA); % factory.add_repairman
                \path[arc,dashed] (rfA) to (sA);
                \path[arc,dashed] (fC) to (rsC);
                \path[arc] (rsC) to (rfC);
                \path[arc,dashed] (rfC) to (sC);
                \path[arc,dashed] (rfC) to (sD);
                \path[arc] (fD) to (rfD);
                \path[arc,dashed] (rfD) to (sD);

                % Start of sim arcs
                \path[draw,thick,->, line join=round, decorate, decoration={
                        zigzag,
                        segment length=8,
                        amplitude=1,post=lineto,
                        post length=2pt
                    }]  (sos1) -- (sA);
                \path[draw,thick,->, line join=round, decorate, decoration={
                        zigzag,
                        segment length=8,
                        amplitude=1,post=lineto,
                        post length=2pt
                    }]  (sos2) -- (eos);
                \path[draw,thick,->, line join=round, decorate, decoration={
                        zigzag,
                        segment length=8,
                        amplitude=1,post=lineto,
                        post length=2pt
                    }]  (sos3) -- (rsA);
            \end{tikzpicture}
            \caption{The eventgraph of the simulation.}\label{fig:event-graph}
        \end{figure}

        \subsection{The event handlers}
            Lorem ipsum dolor sit amet, consetetur sadipscing elitr, sed diam nonumy eirmod
            tempor invidunt ut labore et dolore magna aliquyam erat, sed diam voluptua. At
        \subsection{Performance measures}
            vero eos et accusam et justo duo dolores et ea rebum. Stet clita kasd gubergren,
        \subsection{State}
            no sea takimata sanctus est Lorem ipsum dolor sit amet.
        
    

    \section{Input analysis}
        Lorem ipsum dolor sit amet, consetetur sadipscing elitr, sed diam nonumy eirmod
        tempor invidunt ut labore et dolore magna aliquyam erat, sed diam voluptua. At
        vero eos et accusam et justo duo dolores et ea rebum. Stet clita kasd gubergren,
        no sea takimata sanctus est Lorem ipsum dolor sit amet.


    \section{Experiments}
    \subsection{Set up}
        Lorem ipsum dolor sit amet, consetetur sadipscing elitr, sed diam nonumy eirmod
        tempor invidunt ut labore et dolore magna aliquyam erat, sed diam voluptua. At

    \subsection{Results}
        vero eos et accusam et justo duo dolores et ea rebum. Stet clita kasd gubergren,
        no sea takimata sanctus est Lorem ipsum dolor sit amet.
    
    
    \section{Conclusions}
        Lorem ipsum dolor sit amet, consetetur sadipscing elitr, sed diam nonumy eirmod
        tempor invidunt ut labore et dolore magna aliquyam erat, sed diam voluptua. At
        vero eos et accusam et justo duo dolores et ea rebum. Stet clita kasd gubergren,
        no sea takimata sanctus est Lorem ipsum dolor sit amet.
    

    \section{Appendix}
        Lorem ipsum dolor sit amet, consetetur sadipscing elitr, sed diam nonumy eirmod
        tempor invidunt ut labore et dolore magna aliquyam erat, sed diam voluptua. At
        vero eos et accusam et justo duo dolores et ea rebum. Stet clita kasd gubergren,
        no sea takimata sanctus est Lorem ipsum dolor sit amet.
    

\end{document}
