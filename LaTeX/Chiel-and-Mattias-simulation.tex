\documentclass{article}

\title{Simulation assignment}
\author{Chiel ten Brinke (3677133) and Mattias Beimers (3672565)}
\date{\vspace{-3ex}}

\usepackage{amsmath}
\usepackage{amssymb}
\usepackage{amsthm}
\usepackage{enumerate}
\usepackage{caption}
\usepackage{subcaption}
\usepackage{tikz}
\usepackage{float}
\usepackage[margin=1.75in]{geometry}
\usepackage[parfill]{parskip}

\usetikzlibrary{arrows,shapes,decorations.pathmorphing}

\newcommand{\Break}{\State \textbf{break}}
\renewcommand{\qed}{\hfill \ensuremath{\square}}

\newcommand{\calC}{\ensuremath{\mathcal{C}}}
\newcommand{\calCC}{\ensuremath{\mathcal{C'}}}
\newcommand{\reduc}{\ensuremath{\leq_P} }

\newtheorem{lemma}{Lemma}[section]

\begin{document}
\tikzstyle{vertex}=[circle,very thick,draw=black,fill=black!25,minimum size=35pt,inner sep=0pt,align=center,text width=30pt,font=\scriptsize]
\tikzstyle{fakeVertex}=[circle,minimum size=1pt,inner sep=0pt]
\tikzstyle{edge} = [draw,thick,-]
\tikzstyle{arc} = [draw,thick,->]
\tikzstyle{weight} = [font=\scriptsize]

\maketitle


\section{Problem description}
We present a simulation study for the buffer- and batch-size design of a production line of DVD's.
Moreover, we analyze improvements for the line, by, among others, finding the best buffer- and bacth-sizes and alleviating the impact of bottlenecks.

The production process of DVD's consists of the following steps:
\begin{enumerate}[A.]
    \item Injection molding
    \item Dye coating and drying
    \item Sputtering, lacquer coating, drying
    \item Printing and finishing 
\end{enumerate}
The machines have limited buffer space, and step C is done with batch processing.
Machines can break down, possibly causing them to be out of production for a while,
and discarding DVD's they were producing. Also some machines need cleaning or replacing of ink.
These problems are solved by repairmen, working in shifts.

\section{Assumptions}
These are some of our assumptions:
\begin{enumerate}
    \item
        Machine A discards any DVD's that is being produced at the event of a breakdown.
    \item
        When machine A breaks, sometimes there are two ppl repairing it.
        We assume that this happends half of the time.
        This way we are not likely to assume too little.
    \item
        Machines C and D are expected to start with cleaning or ink replacing after the (batch of) DVD's is finished.
    \item
        We assume that a repairman finishes repairing the machine before going home.
    \item
        We start the simulation at 6 o'clock in the morning.
\end{enumerate}

\section{Analysis of the problem}
We want to solve this problem by performing a simulation study. % ???


\section{The model}
\subsection{State}
The factory has a set of state variables, and each of the machines also has some state variables of his own.
The state of the factory consists of the current time, whether it is day or night and the amount of repairman available.
Of course, the state also contains a list of machines, each of these machines have the following state variables:
The status of the machine, this is used to see if the machine is busy producing DVD's, if it is currently not doing anything, or if it's broken or being repaired (possibly with two repairmen).
It also maintains whether or not it has discarded it's latest DVD\@.
Since each machine is assigned his own buffer (with exception of machines A - for them two machines share a buffer), it also maintains the amount of DVD's in the buffer.

\subsection{Performance measures}
\label{performance_measures}
We measure the total production of DVD's, the average production per DVD,
and the average throughput.
Apart from that, we measure some more statistics per machine, to look for bottlenecks:
The number of breakdowns, the total amount of DVD's a machine has produced,
the total amount of discarded DVD's and the time the machine is down,
busy producing, or doing nothing (idle).

\subsection{The eventgraph}
This is our event graph.
\begin{figure}[H]
    \centering
    \begin{tikzpicture}[auto,swap]
        % The machine start and finish events
        \node[vertex] (sA) at (-2,0) {MA start};
        \node[vertex] (fA) at (1,0) {MA finish};
        \node[vertex] (sB) at (-2,-2) {MB start};
        \node[vertex] (fB) at (1,-2) {MB finish};
        \node[vertex] (sC) at (-2,-5) {MC start};
        \node[vertex] (fC) at (1,-5) {MC finish};
        \node[vertex] (sD) at (-2,-7) {MD start};
        \node[vertex] (fD) at (1,-7) {MD finish};

        % Buffer events
        \node[vertex] (assembly) at (-0.5,-3.5) {Put in crate};

        % The breakdown events
        \node[fakeVertex] (sos3) at (-5,-2.2) {};
        \node[vertex] (rfA) at (-4,1) {MA finishes repairing};
        \node[vertex] (rsA) at (-4.5,-1.1) {MA starts repairing};
        \node[vertex] (rfC) at (-4,-4.6) {MC finishes cleaning};
        \node[vertex] (rsC) at (-4,-6.6) {MC starts cleaning};
        \node[vertex] (rfD) at (-0.5,-8.5) {MD finishes ink replacing};

        % The start- and end- of simulation events
        \node[fakeVertex] (sos1) at (-2.5,-1) {};
        \node[fakeVertex] (sos2) at (3,-7.5) {};
        \node[vertex] (eos) at (3.5,-6.5) {End of sim};

        % Arcs for machine start and finish events
        %\node[fakeVertex] (radius4) at (-2.0,1.1) {};
        %\node[fakeVertex] (radius5) at (-1.8,0.6) {};
        %\node[fakeVertex] (radius6) at (-2.2,0.6) {};
        %\path[edge,bend left=60,dashed] (radius6) to (radius4);
        %\path[arc,bend left=60,dashed] (radius4) to (radius5);
        \path[arc,bend left=10] (sA) to (fA);
        \path[arc,bend left=10,dashed] (fA) to (sA);
        \path[arc,dashed] (fA) to (sB); % buffer.add_product
        \path[arc,bend left=10] (sB) to (fB);
        \path[arc,bend left=10,dashed] (fB) to (sB);
        \path[arc,bend left=10] (sC) to (fC);
        \path[arc,bend left=10,dashed] (fC) to (sC);
        \path[arc,dashed] (fC) to (sD); % buffer.add_product
        \path[arc,bend left=10] (sD) to (fD);
        \path[arc,bend left=10,dashed] (fD) to (sD);
        %\path[arc] (fD) to (eos);

        % Arcs for buffer events
        \path[arc,dashed] (sB) to (sA); % buffer.remove_product
        \path[arc,dashed] (fB) to (assembly); % buffer.add_product
        \node[fakeVertex] (radius1) at (-0.5,-2.6) {};
        \node[fakeVertex] (radius2) at (-0.4,-2.9) {};
        \node[fakeVertex] (radius3) at (-0.6,-2.9) {};
        \path[edge,bend left=60,dashed] (radius3) to (radius1);
        \path[arc,bend left=60,dashed] (radius1) to (radius2);
        \path[arc,dashed] (assembly) to (sC); % buffer.add_product
        \path[arc,dashed] (sC) to (sB); % buffer.remove_product
        \path[arc,dashed] (sD) to (sC); % buffer.remove_product

        % Arcs for breakdown events
        \path[arc,bend left=15] (rsA) to (rfA);
        \path[arc,bend left=15,dashed] (rfA) to (rsA); % factory.add_repairman
        \path[arc,dashed] (rfA) to (sA);
        \path[arc,dashed] (fC) to (rsC);
        \path[arc] (rsC) to (rfC);
        \path[arc,dashed] (rfC) to (sC);
        \path[arc,dashed] (rfC) to (sD);
        \path[arc] (fD) to (rfD);
        \path[arc,dashed] (rfD) to (sD);

        % Start of sim arcs
        \path[draw, thick, ->, line join=round, decorate, decoration=
            {zigzag, segment length=8, amplitude=1,post=lineto, post length=2pt
            }]  (sos1) -- (sA);
        \path[draw, thick, ->, line join=round, decorate, decoration=
            {zigzag, segment length=8, amplitude=1,post=lineto, post length=2pt
            }]  (sos2) -- (eos);
        \path[draw, thick, ->, line join=round, decorate, decoration=
            {zigzag, segment length=8, amplitude=1,post=lineto, post length=2pt
            }]  (sos3) -- (rsA);
    \end{tikzpicture}
    \caption{The eventgraph of the simulation.}\label{fig:event-graph}
\end{figure}

\subsection{The event handlers}
% Updating state - measure performance, schedule new events.
\begin{enumerate}
    % The machine start and finish events
    \item Machine start:
        This is the event where the machine starts producing his DVD (or batch of DVD's).
        It removes a DVD from the previous buffer (except for machine A of course, it creates a new DVD), it updates it's status to busy, and then schedules the machine end event.
        Removing an item from the previous buffer can possibly schedule a machine start event, if it was doing nothing because the buffer was full.
    \item Machine finish:
        This is the event where the machine finishes producing his DVD (or batch of DVD's).
        It sets it's status to idle
        % Specials for each machine

    % Buffer events
    \item Put in crate:

    % The breakdown events
    \item MA starts repairing:
    \item MA finishes repairing:
    \item MC starts cleaning:
    \item MC finishes cleaning:
    \item MD finishes ink replacing:

    % EOS
    \item End of simulation:
        This event ends the simulation, it fires either after a specified duration, or when the user stops the simulation manually.
        
\end{enumerate}


\section{Input analysis}
Some parameters are exact, others have a certain degree of randomness.
For the latter ones, we present here how the data collection (i.e.\ generation) is implemented in the simulation.

Some input spaces are given in the form of a theoretical probability distribution.
Others are given in the form of an empirical probability distribution.
For dealing with empirical distributions, we pick a random number from the sorted empirical data, then pick a number between this number and the next number with uniform probability, i.e.\ we interpolate the empirical distribution linearly.
This way we do not have to care about catching the empirical data in a theoretical probability distribution.
In table~\ref{table:input_table} we have listed the kinds of input data we have for the several machine parameters.

\begin{table}[h]
    \begin{tabular}{|l|l|l|l|p{4cm}|}
    \hline
                    & Machine 1         & Machine 2         & Machine 3                     & Machine 4 \\ \hline
    uptime          & $\exp(480*60)$    &                   & 3\% of batches                & \parbox{5cm}{40\%, 40\%, 20\% uniformly\\ after producing resp.\\ 200, 200±1, 200±2 dvd's} \\ \hline
    repairtime      & $\exp(120*60)$    &                   & exact 5 min                   & $\mathrm{normal}(15*60, 60)$ \\ \hline
    production time & empirical         & empirical         & $\exp(10) + \exp(7) + 3*60$   & empirical \\ \hline
    dvd discards    &                   & 2\% of the dvd's  &                               & \\ \hline
    \end{tabular}
    \caption {Input distributions}
    \label{table:input_table}
\end{table}

%Machine 1:
    %uptime: exp distr 480min
    %repairtime: exp distr 120min
    %production time: empirical distribution

%Machine2:
    %dvd discard: 2\% of the dvd's uniformly
    %production time: empirical distribution

%Machine3:
    %uptime: 3\% of the batches uniformly
    %production time: exp 10sec / dvd + exp 7 sec / dvd + 3min

%Machine4:
    %uptime: resp. 40\%, 40\%, 20\% uniformly after resp. 200, 200±1, 200±2
    %repairtime: normal mean 15min std 1min
    %production time: empirical distribution





\section{Experiments}
% TODO:
% - arguments for running time of choice (steady state etc)
% - input analysis
%
\subsection{Set up}
Our goal is to give insight in possible improvements in the production process of the dvd production line with respect to changes in the buffersizes and the batchsize of the machines.
To do so, we will mainly investigate the performance measures mentioned in section~\ref{performance_measures}, and see wether they can coexist optimally or there is a trade-off involved.
While having the simulation at hand anyway, we can also have a look for possible bottlenecks caused by things other than the configuration of the buffersizes and the batchsize.

We will choose various configurations for the buffersizes and the batchsize, and run the simulation several times for each of the configurations.

For each configuration:
    run simulation and compute performance measures and confidence interval
    conclude which configurations provide significant improvements


As confidence level $1 - \alpha$ we choose the fairly standard value of $0.95$.
Suppose that we run the simulation, for each configuration, $n$ times, resulting in $n$ sample values, for each performance measure, with mean $\bar{x}$ and deviation $s$.

Then $P(-c \leq T \leq c) = 0.95 $ with $T = \frac{\bar{x} - \mu}{s/\sqrt{n}}$ (student t distribution) and $c = \Phi ^{-1} (1 - \frac{\alpha}{2})= 1.96$.
From this it follows that the confidence interval can be computed as
\[ (\bar{x} - \frac{cs}{\sqrt{n}}, \bar{x} + \frac{cs}{\sqrt{n}}) =
(\bar{x} - \frac{1.96s}{\sqrt{n}}, \bar{x} + \frac{1.96s}{\sqrt{n}}). \]


\subsection{Results}
vero eos et accusam et justo duo dolores et ea rebum. Stet clita kasd gubergren,
no sea takimata sanctus est Lorem ipsum dolor sit amet.


\section{Conclusions}
Lorem ipsum dolor sit amet, consetetur sadipscing elitr, sed diam nonumy eirmod
tempor invidunt ut labore et dolore magna aliquyam erat, sed diam voluptua. At
vero eos et accusam et justo duo dolores et ea rebum. Stet clita kasd gubergren,
no sea takimata sanctus est Lorem ipsum dolor sit amet.


\section{Appendix}
Lorem ipsum dolor sit amet, consetetur sadipscing elitr, sed diam nonumy eirmod
tempor invidunt ut labore et dolore magna aliquyam erat, sed diam voluptua. At
vero eos et accusam et justo duo dolores et ea rebum. Stet clita kasd gubergren,
no sea takimata sanctus est Lorem ipsum dolor sit amet.


\end{document}
